\section{01 \textbar\ Getting Started}

This document is the practical entry point into RCX-$\pi$.
After reading this section you should be able to:

\begin{itemize}
    \item clone the repository and enter the working directory,
    \item run the full demo + test suite,
    \item experiment interactively using the minimal REPL,
    \item understand where to go next in the documentation.
\end{itemize}

RCX-$\pi$ is intentionally small --- you can inspect and understand the full
core in one sitting. This section gets your hands on a living instance quickly
rather than only reading abstract theory.

%----------------------------------------------------------
\subsection{1. Clone the Repository}

\begin{verbatim}
git clone https://github.com/jabramsja/rcx-pi-core.git
cd rcx-pi-core/WorkingRCX
\end{verbatim}

The working tree contains:

\begin{verbatim}
WorkingRCX/
  rcx_pi/            # Core RCX-π package
  demo_rcx_pi.py     # End-to-end demonstration
  example_numbers.py # Peano / arithmetic examples
  example_rcx.py     # Closure / swap / rotate examples
  example_higher.py  # Higher-level helpers (factorial, map/sum)
  bench_rcx.py       # Micro benchmark for reductions
  repl_rcx.py        # Minimal interactive REPL
  run_all.py         # Unified test + demo runner
  tests/             # Validation scripts
\end{verbatim}

No installation step is required --- the engine runs locally in-place.

%----------------------------------------------------------
\subsection{2. Python Environment}

A standard CPython $\geq$ 3.10 is recommended.

\begin{verbatim}
python3 --version
\end{verbatim}

On macOS and Linux the usual invocation is:

\begin{verbatim}
python3 run_all.py
\end{verbatim}

(Replace \verb|python3| with \verb|python| if your system maps it to Python~3.)

%----------------------------------------------------------
\subsection{3. Run the Full Demo \& Test Suite}

From inside \verb|WorkingRCX/|:

\begin{verbatim}
python3 run_all.py
\end{verbatim}

Expected output structure:

\begin{verbatim}
=== RCX-π Full Test & Demo Runner ===

>>> Running demo_rcx_pi.py
[OK]

>>> Running example_numbers.py
[OK]

>>> Running test_*.py
[OK]

=== End of RCX-π test suite ===
\end{verbatim}

If a failure occurs, the runner shows the failing script and a traceback.
You can execute a single component directly:

\begin{verbatim}
python3 demo_rcx_pi.py
python3 example_rcx.py
python3 tests/test_numbers.py
\end{verbatim}

%----------------------------------------------------------
\subsection{4. Interactive REPL}

Launch:

\begin{verbatim}
python3 repl_rcx.py
\end{verbatim}

You should see:

\begin{verbatim}
=== RCX-π REPL ===
Type 'help' for commands, 'quit' to exit.
rcx>
\end{verbatim}

Example session:

\paragraph{Peano Numbers}

\begin{verbatim}
rcx> num 5
motif: μ(μ(μ(μ(μ(μ())))))
int:   5
\end{verbatim}

\paragraph{Pairs \& structural programs}

\begin{verbatim}
rcx> pair 2 5
motif: μ(μ(μ(μ())), μ(μ(μ(μ(μ(μ()))))))
pair:  (2, 5)

rcx> swap 2 5
...
reduced => (5, 2)

rcx> rot 2 5 7
...
reduced => (5, 7, 2)
\end{verbatim}

\paragraph{Meta classification}

\begin{verbatim}
rcx> classify pair 2 5
motif:    μ(...)
tagged:   μ(tag, payload)
pretty:   <value> (2, 5)
\end{verbatim}

\paragraph{Safety Probes}

\begin{verbatim}
rcx> safe num 5
is_self_host_safe: True

rcx> safe pair 2 5
is_self_host_safe: True
\end{verbatim}

\verb|help| in the REPL lists available commands.

%----------------------------------------------------------
\subsection{5. Optional Shell Shortcuts}

If you frequently enter the working directory, you may define aliases.

Example for \verb|~/.zshrc|:

\begin{verbatim}
alias wrx='cd ~/Desktop/RCX_X/RCXStack/RCXStackminimal/WorkingRCX'
alias gl='git log --oneline --graph --decorate --all'
\end{verbatim}

Reload:

\begin{verbatim}
source ~/.zshrc
wrx   # jump directly to WorkingRCX/
gl    # view prettified git history
\end{verbatim}

Adjust the path in \verb|wrx| to match your environment.

%----------------------------------------------------------
\subsection{6. Where to Go Next}

Once \verb|run_all.py| passes with \verb|[OK]| and you have played inside the
REPL, move deeper:

\begin{itemize}
    \item \textbf{00-overview} --- motivation and conceptual framing
    \item \textbf{02-core-structures} --- motif representation and evaluation
    \item \textbf{03-program-library} --- structural programs and composition
\end{itemize}

You now have a live RCX-$\pi$ environment and the shortest path to experiments.
From here the system opens outward: self-hosting, growth laws, meta tags,
recursion scaffolds and the eventual RCX higher manifolds.