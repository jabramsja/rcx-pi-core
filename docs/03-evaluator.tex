\section{03 \textbar\ The Evaluator and Reduction Rules}

RCX-$\pi$ has no opcodes, no bytecode, and no virtual machine.  
Computation is enacted by a \textbf{pure reduction engine} that rewrites
motif trees according to shape alone.

\[
\text{Motif} \xrightarrow{\text{rewrite}} \text{Motif}
\]

There is no semantic substrate beneath it — \emph{shape is both data and execution.}

%----------------------------------------------------------
\subsection{Purpose of the Evaluator}

Given a motif \(M\), the evaluator repeatedly applies rewrite rules until
no further rule matches:

\[
\text{reduce}(M) =
\begin{cases}
M' & \text{if } M \Rightarrow M' \\
M  & \text{otherwise}
\end{cases}
\]

Where \(\Rightarrow\) indicates a single structural contraction.

Evaluation halts when the motif reaches \textbf{normal form} (stable geometry).

%----------------------------------------------------------
\subsection{Rewrite Semantics}

All rules in RCX-$\pi$ follow the same pattern:

\[
\mu(\text{pattern}, \text{arguments}) \Rightarrow \text{replacement}
\]

Arguments are \emph{subtrees} — no variable environments, no symbol table,
no call frames. Matching is purely geometric.

Illustrative rule:

\[
\mu(\text{succ},\mu()) \Rightarrow \mu(\mu()) \equiv 1
\]

Note: numerals are not primitive — they are \(\mu\)-chains.

%----------------------------------------------------------
\subsection{Core Numeric Reductions}

Peano naturals arise by nested \(\mu\)-application:

\[
0 := \mu()
\qquad
n+1 := \mu(n)
\]

Reduction rules:

\[
\text{pred}(\mu(n)) \Rightarrow n
\qquad\qquad
\text{succ}(n) := \mu(n)
\]

Addition and multiplication emerge from structure rather than arithmetic:

\[
a+b := \mu(a,b)
\quad
\Rightarrow\quad
\mu^{a+b}()
\]

\[
a\times b := \mu(a,b,\text{mult})
\quad\Rightarrow\quad
\mu^{ab}()
\]

The evaluator does not "compute numbers" —  
it \textbf{folds geometry into canonical chains}.

%----------------------------------------------------------
\subsection{Program Activation}

Programs are motifs. Function application is the moment structure aligns:

\[
\mu(\text{swap},\mu(x,y)) \Rightarrow \mu(y,x)
\]
\[
\mu(\text{rot},\mu(x,y,z)) \Rightarrow \mu(y,z,x)
\]

No syntax dictates "call" — activation is a physical configuration.

\[
\text{execution} := \text{shape rearrangement}
\]

%----------------------------------------------------------
\subsection{Reduction Strategy}

\[
\text{reduce}(M) =
M \Rightarrow M_1 \Rightarrow M_2 \Rightarrow \dots \Rightarrow M_n
\]

Like gravity settling blocks into place,
evaluation is the relaxation of tension in structure.

%----------------------------------------------------------
\subsection{Determinism and Confluence}

Goal: \textbf{confluence} — different rewrite paths produce same result.

Sometimes motifs create divergent normal forms — a feature, not a bug.
These behaviors will feed later RCX research:

\begin{itemize}
\item reversible and bidirectional rewrites
\item chaotic activation forests
\item self-observing reductions
\item evolving internal rule sets
\end{itemize}

The evaluator is minimal by design — \textit{simplicity fuels emergence}.

%----------------------------------------------------------
\subsection{Example Reduction Trace}

\[
\text{swap}(2,5) = \mu(\text{swap},\mu(2,5))
\]

\[
\mu(\text{swap},\mu(a,b)) \Rightarrow \mu(b,a) \Rightarrow (5,2)
\]

Execution history is visible as living structure.

Reduction is not hidden --- it \textbf{is the runtime}.

%----------------------------------------------------------
\subsection{Summary}

\begin{itemize}
\item The evaluator rewrites motifs until no rules apply
\item Structure alone defines computation — no syntax layer
\item Programs and data share the same form
\item Activation is geometric: place forms together and reduce
\item Execution traces are motifs changing shape in the open
\end{itemize}

RCX-$\pi$ is computation without instructions.  
The engine is geometry.