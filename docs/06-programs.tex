\section{06 \textbar\ Programs, Closures, and Activation in RCX-$\pi$}

A remarkable property of RCX-$\pi$ is that \emph{programs are not separate
entities}. A function, a value, a list, and a closure all inhabit the same
universe:

\[
\text{everything is a motif.}
\]

There are no keywords, no opcodes, no call stack. A program is simply a
motif shaped so that, when placed against data, a reduction rule fires.
Computation is what happens when geometry aligns.

\subsection*{Motifs as Programs}

A \textbf{closure} in RCX-$\pi$ is just a motif representing a structural
transformation. It becomes a program only when data is attached.

\[
\text{program} := \mu(\text{pattern},\dots)
\]

\[
\text{activation} := \mu(\text{program},\text{data})
\Rightarrow \text{result}
\]

There is no symbolic difference between:

\[
\underbrace{\text{value}}_{\mu(\mu(\mu()))}
\qquad
\underbrace{\text{program}}_{\mu(\dots)}
\qquad
\underbrace{\text{activation}}_{\mu(\text{program},\text{data})}
\]

All three are motifs — only structure distinguishes the role they play.

%----------------------------------------------------------
\subsection{Example Closures}

RCX-$\pi$ ships with several canonical shape-transformers:

\[
\text{swap}_{xy} : (x,y) \mapsto (y,x)
\]
\[
\text{dup}_x : (x,y) \mapsto (x,x)
\]
\[
\text{rot}_{xyz} : (x,y,z) \mapsto (y,z,x)
\]

All of these are encoded without syntax or variables. They \emph{are
motifs}, not code about motifs.

\[
\text{swap}_{xy}
  = \mu(\dots)   % not expanding here for brevity
\]

\[
\text{rot}_{xyz}
  = \mu(\dots)
\]

Their representations live in \texttt{rcx\_pi/programs.py}.

%----------------------------------------------------------
\subsection{Activation is Geometry}

To evaluate \(\text{swap}(2,5)\) we do not "call a function".
Instead we \emph{grow a tree}:

\[
\mu(\text{swap},\mu(2,5))
\]

Reduction rules inside the evaluator recognize the pattern:

\[
\mu(\text{swap},\mu(a,b)) \Rightarrow \mu(b,a)
\]

\[
\Rightarrow (5,2)
\]

There is no stack or environment. Execution is literal form-shifting —
like a piece of origami folding into meaning.

%----------------------------------------------------------
\subsection{Activation Walkthrough Examples}

\paragraph{Swap a Pair}
\[
\text{swap}(2,5)
\Rightarrow (5,2)
\]

\paragraph{Duplicate First Element}
\[
\text{dup}(2,5)
\Rightarrow (2,2)
\]

\paragraph{Rotate Triple}
\[
\text{rot}(2,5,7)
\Rightarrow (5,7,2)
\]

Every closure is simply a structural rewrite rule waiting to fire.

%----------------------------------------------------------
\subsection{Programs as Data}

Because closures are motifs, they can be:

\begin{itemize}
\item passed as values
\item stored inside other motifs
\item sent through meta-classifiers
\item inspected, tagged, and rearranged
\item activated recursively on their own structure
\end{itemize}

This is the seed of RCX self-hosting: the interpreter could eventually
be expressed in the same motif language it executes.

\[
\text{future: }
\mu(\text{eval}, M) \Rightarrow \text{reduce}(M)
\]

When the evaluator itself becomes a motif, RCX-$\pi$ becomes a closed
world capable of reflection.

%----------------------------------------------------------
\subsection{Design Consequences}

\begin{itemize}
\item No syntax or instruction set needed
\item Functions and data unify under one form
\item Execution is structural alignment, not interpretation
\item Programs are visible objects — you can \emph{look at a function}
\item Higher-order behavior emerges naturally
\end{itemize}

\[
\text{Computation = Geometry + Rewrite Rules}
\]

RCX-$\pi$ is not coded — it is \emph{grown}.

%----------------------------------------------------------
\subsection{Next: Higher-Level Behavior (factorials, maps, folds)}

With closures and activation understood, we can now climb one step up:

\[
\text{structure} \Rightarrow \text{programs}
\Rightarrow \text{patterns of computation}
\]

In the next chapter we construct factorial, summation, \texttt{map}, and
other classical patterns \emph{using only motifs}.