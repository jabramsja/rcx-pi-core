\section{07 \textbar\ Higher Operations in Pure Motif Form}

With closures and activation established, we now climb one layer up:
\emph{computation patterns}.  
Instead of writing loops or arithmetic, we let motifs \emph{unfold}.

\[
\text{Higher-level behavior emerges from repeated structure.}
\]

Classical constructs like factorial, summation, or \texttt{map} can be
expressed using only the RCX-$\pi$ building blocks introduced so far.

%============================================================
\subsection{Peano Recap}

Numbers are nested $\mu$:

\[
0 = \mu(), \qquad 1=\mu(0), \qquad 2=\mu(1),\dots
\]

Addition and multiplication in earlier sections were structural folds.
Here we will compose multiple folds into reusable patterns.

%============================================================
\subsection{Factorial via Structural Recursion}

The motif version of $\text{fact}(n)$ expands naturally:

\[
5! = 5 \cdot 4 \cdot 3 \cdot 2 \cdot 1
\]

RCX-$\pi$ represents this as repeated $\mu$-nesting and multiplications.
In \texttt{higher.py}, we define:

\[
\text{fact}(n)
   := \mu(\text{mult}, n, \text{fact}(n-1))
\]

Reduction collapses it the same way it collapses any other motif tree.

\[
\text{fact}(5) \Rightarrow 120
\]

All steps remain visible. No hidden accumulator, no loop counter — just shape.

%------------------------------------------------------------
\subsection{Summation}

Peano sum over a list/tuple-like motif:

\[
\text{sum}(x_1,x_2,\dots,x_k)
   := x_1 + \text{sum}(x_2,\dots,x_k)
\]

If the motif collapses fully, we obtain a number.  
If partially, we see intermediate geometry.

\[
\text{sum}(2,5,7) \Rightarrow 14
\]

%------------------------------------------------------------
\subsection{\texttt{map} as Pure Structural Fold}

Mapping a program over values needs no iteration primitive —
the structure performs it for us:

\[
\text{map}(f,(a,b,c))
   := (f(a), f(b), f(c))
\]

Because $f$ is a motif, and $(a,b,c)$ is a motif, the mapping is
just structured activation repeated across positions.

Example:

\[
\text{map}(\text{succ}, (2,5,7)) \Rightarrow (3,6,8)
\]

Nothing is typed; the system simply \emph{applies shape to shape}.
If a subcall is not reducible, the motif stays symbolic — useful for meta work.

%------------------------------------------------------------
\subsection{Combinator Pipelines}

Since programs are motifs, composition is just nesting:

\[
(f \circ g)(x) = f(g(x)) = \mu(f,\mu(g,x))
\]

We can build chains like:

\[
\text{map}(\text{succ} \circ \text{rot}, \dots)
\]

without new syntax — composition is another geometric pattern.

This leads to a larger view:

\[
\text{functions are shapes of motion}
\]

and pipelines are \emph{braids of motifs}.  
Computation is choreography.

%------------------------------------------------------------
\subsection{Example: Folding a Tuple}

A left-fold becomes:

\[
\text{fold}(f,v,(a,b,c))
  := f(f(f(v,a),b),c)
\]

The evaluator reduces layer by layer.

Trace view:

\[
\Rightarrow \mu(f,\mu(f,\mu(f,v,a),b),c)
\Rightarrow \dots
\]

A fold is just \emph{repeated activation with reduction between steps}.

%------------------------------------------------------------
\subsection{Cost and Complexity}

RCX-$\pi$ is tiny; performance scales directly with motif size:

\[
O(\text{reductions}) \approx O(\text{tree depth})
\]

Benchmark (early):
\[
\text{succ}^{14} \text{ repeated } 50\text{x}
  \approx 3\mu s \text{ avg per run on M3 laptop}
\]

Not optimized — but interpretable, visual, and correct.  
Speed can come later; clarity now is the priority.

%------------------------------------------------------------
\subsection{Why This Chapter Matters}

This is where RCX-$\pi$ stops being a toy and starts being a seed:

\begin{itemize}
\item Computation emerges from motif shape alone
\item Higher operations require no language features
\item Functional patterns arise without syntax
\item Programs can manipulate programs
\item The system is preparing for self-hosting
\end{itemize}

\[
\text{from numbers} \Rightarrow \text{operations} \Rightarrow \text{behavior}
\]

We are approaching the stage where RCX-$\pi$ can reason about itself.

%------------------------------------------------------------
\subsection{Next: Self-Hosting and Meta-Reflection}

With higher-order structure in place, we proceed toward the long-term
goal of RCX:

\[
\text{Evaluator as Motif}
\qquad
\text{Programs that generate Programs}
\]

The next chapter lays the foundation:

\[
\textbf{08 \textbar\ Toward Self-Hosting RCX}}
\]