\section{00 \textbar\ RCX-$\pi$ Core Overview}

RCX-$\pi$ is a \textbf{tiny structural engine} built entirely on one primitive constructor:

\[
\mu(\;\cdot\;)
\]

Everything arises from this single form — no types, no bytecode, no opcodes.
Numbers, programs, meta-operations, and even evaluators are \emph{motifs}.
Computation occurs by \textbf{structural rewriting} rather than executing
instructions.

RCX-$\pi$ is a minimal, executable slice of RCX theory. It demonstrates how
computation can emerge from structure alone, and is deliberately small enough
that a human can inspect and evolve it directly.

%----------------------------------------------------------
\subsection{What is a Motif?}

A \textbf{motif} is a tree:

\[
\mu(\;\mu(\;\mu(\dots)\;)\;)
\]

The empty motif is:

\[
\mu()
\]

Peano naturals are nested applications of $\mu$:

\begin{center}
\begin{tabular}{c|c}
\textbf{Number} & \textbf{Motif} \\
\hline
0 & $\mu()$ \\
1 & $\mu(\mu())$ \\
2 & $\mu(\mu(\mu()))$ \\
5 & $\mu$ applied 6 times \\
\end{tabular}
\end{center}

There is no semantic type system.
A number, a closure, a tuple, a program — all are motifs.
Their identity comes only from \textbf{shape} and \textbf{reduction behavior}.

%----------------------------------------------------------
\subsection{Evaluation}

Reduction is defined in \verb|evaluator.py|.
There is no instruction set; instead RCX-$\pi$ computes by structural pattern
collapse.

Computation is visible and inspectable as geometry.

Example:

\[
\text{pred}(\text{succ}(0))
\]

reduces structurally — nothing is executed, only folded.

%----------------------------------------------------------
\subsection{Programs as Motifs}

Programs are not separate entities.
A closure \emph{is} a motif that expects activation.

\begin{center}
\begin{tabular}{l|l}
\textbf{Name} & \textbf{Meaning} \\
\hline
\verb|swap_xy_closure| & $(x,y) \rightarrow (y,x)$ \\
\verb|dup_x_closure| & $(x,y) \rightarrow (x,x)$ \\
\verb|rotate_xyz_closure| & $(x,y,z) \rightarrow (y,z,x)$ \\
\end{tabular}
\end{center}

Activation is structural growth:

\[
\text{closure} \;+\; \text{data} \; \Rightarrow \; \mu(\text{closure},\text{data})
\]

then reduction collapses the geometry into a result.
No VM stack — computation is origami.

%----------------------------------------------------------
\subsection{Activation Example}

During evaluation you can watch the motif twist and collapse.
Execution becomes visual rather than symbolic — instructions are replaced
by \emph{shape transformations}.

%----------------------------------------------------------
\subsection{Structural Classification Layer}

RCX-$\pi$ includes a minimal \textbf{meta-classifier} that inspects motifs and
tags them as:

\begin{center}
\begin{tabular}{l|l}
\textbf{Label} & \textbf{Meaning} \\
\hline
\verb|value| & numeric/atomic \\
\verb|program| & closure awaiting activation \\
\verb|mixed| & partial application \\
\verb|struct| & generic composite motif \\
\end{tabular}
\end{center}

Example classification:

\[
\text{pair}(2,5)
\;\Rightarrow\;
\langle \text{value} \rangle (2,5)
\]

Motifs become reflective — structure is visible to itself.

%----------------------------------------------------------
\subsection{Pairs, Triples, and Higher Arity}

Tuples require no new rules — they are just nested motifs.
Closures operate by rearranging structure:

\[
(x,y,z) \mapsto (y,z,x)
\]

Everything is \emph{fold, reorder, collapse}.

Structure \textbf{is} the computation.

%----------------------------------------------------------
\subsection{Why This Matters}

Traditional systems rely on:

\begin{itemize}
\item opcodes
\item call stacks
\item environments
\item interpreters evaluating interpreters
\end{itemize}

RCX-$\pi$ replaces all of this with:

\[
\text{Geometry} + \text{Reduction Rules} = \text{Computation}
\]

Programs are not written — they \textbf{grow}.
Evaluation is physical, spatial, inspectable.

%----------------------------------------------------------
\subsection{Higher-Level Toolkit (v1.2+)}

The engine remains minimal, but a standard library of structural patterns is
emerging:

\begin{center}
\begin{tabular}{l|p{7cm}}
\textbf{Feature} & \textbf{What it does} \\
\hline
\verb|motif_to_int| & collapse Peano motif to Python integer \\
\verb|num(n)| & lift Python integer to motif form \\
\verb|pretty(m)| & pretty-print motif as $(a,b,c)$ \\
\verb|bench()| & micro-benchmark evaluator \\
\verb|higher.py| & factorial, summation, map, tuple ops \\
\verb|self_host.py| & seed of eventual RCX self-boot process \\
\end{tabular}
\end{center}

All helpers are optional — the core depends on none.

%----------------------------------------------------------
\subsection{Pretty Printing}

Raw motifs are dense:

\[
\mu(\mu(\mu(\mu(\dots))))
\]

Pretty printing renders nested arity naturally:

\[
(2,5,7)
\]

allowing large RCX structures to be debugged with clarity.

%----------------------------------------------------------
\subsection{Higher-Order Combinators}

Functional patterns arise from shape alone:

\[
\text{map},\; \text{fold},\; \text{sum},\; \text{factorial}
\]

No mutation, no loops — recursion is pattern propagation.

%----------------------------------------------------------
\subsection{Benchmarks}

A tiny engine with tiny timing.
Motif reduction is light and fast — like a watch-spring.

%----------------------------------------------------------
\subsection{Vision Beyond the Core}

Today RCX-$\pi$ is a \textbf{self-consistent computational seed}.
Tomorrow:

\begin{itemize}
\item typed motifs
\item self-host evaluator
\item evolving closures
\item membrane/lobe growth
\item semantic fold → behavior
\item emergence instead of instruction
\end{itemize}

The core will not grow larger — the universe grows around it.