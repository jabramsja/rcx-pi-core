\section{02 \textbar\ Core Structures of RCX-$\pi$}

This section explains the essential building blocks of computation in RCX-$\pi$.
Everything reduces to a single data form — the motif — and a small set of
structural rules that govern how motifs combine, transform, and collapse.

RCX-$\pi$ does not distinguish between data and code. A number, a pair, a closure,
a program, and a meta-transform are all constructed from identical material.
Computation is the \emph{geometry of motifs}.

%----------------------------------------------------------
\subsection{The Motif Primitive}

All structure is built from a single constructor:

\[
\mu(a_1, a_2, \dots, a_n)
\]

where $n \ge 0$ and each $a_i$ is itself a motif.
There are no alternative node types, tags, or syntactic layers.

\[
\text{Motif} := \mu(M_1, M_2, \dots, M_k)
\]

\textbf{Everything is a tree.}
No values, variables, or functions exist apart from shape.

\begin{itemize}
\item $\mu()$ is the empty motif (Peano zero)
\item $\mu(X)$ is unary structure (successor or embedding)
\item $\mu(X,Y)$ is binary — the seed of pairs/closures
\item Higher arity emerges recursively
\end{itemize}

%----------------------------------------------------------
\subsection{Peano Naturals as Pure Structure}

Integers require no numeric type. They are purely spatial depth.

\[
0 := \mu()
\qquad
1 := \mu(0)
\qquad
2 := \mu(1)
\qquad ...
\]

\[
n := \mu^n(\mu())
\]

Reduction can convert motifs to integers via collapse, but no numeric layer is
required for computation.

%----------------------------------------------------------
\subsection{Pairs and Tuples}

Pairs are not a datatype — just arity-2 motifs:

\[
(x,y) := \mu(x,y)
\]

Triples generalize naturally:

\[
(x,y,z) := \mu(x,y,z)
\]

and arbitrary tuples follow the same pattern. RCX-$\pi$ does not need
type constructors for lists, arrays, or vectors. Structure alone encodes them.

\begin{center}
\begin{tabular}{c|c}
\textbf{Form} & \textbf{Meaning (informal)} \\
\hline
$\mu(a,b)$ & pair $(a,b)$ \\
$\mu(a,b,c)$ & triple $(a,b,c)$ \\
$\mu(a,b,c,d,\dots)$ & n-tuple \\
\end{tabular}
\end{center}

The evaluator does not special-case tuples. Interpretation is purely pattern-based.

%----------------------------------------------------------
\subsection{Closures as Motifs}

A program is simply a motif that, when activated, reduces into a new shape.
There are no lambdas, argument lists, or binding rules. \emph{A closure is
structure awaiting another structure.}

\[
\text{closure}(f,x) = \mu(f,x)
\]

where evaluation rules know how to collapse it.

Example program motifs:

\[
\text{swap}(x,y) \rightarrow (y,x)
\qquad
\text{rot}(x,y,z) \rightarrow (y,z,x)
\]

There is no code vs data barrier. A closure is indistinguishable from data until
reduction triggers semantic behavior.

%----------------------------------------------------------
\subsection{Structural Rewriting}

Computation occurs when motifs match one of the evaluator’s rewrite schemas:

\[
\mu(\text{rule},\text{args}) \Rightarrow \text{reduced motif}
\]

Rules live in \verb|evaluator.py|, forming the ``physics'' of RCX-$\pi$.

Example (conceptual):

\[
\mu(\text{succ},\mu()) \Rightarrow 1
\qquad
\mu(\text{pred},1) \Rightarrow 0
\]

but no numeric primitives exist — all are patterns over motifs.

%----------------------------------------------------------
\subsection{Activation}

To call a program, we grow a motif:

\[
\mu(\text{closure},\text{data}) \Rightarrow \text{reduction cascade}
\]

Execution is spatial. There is no instruction pointer.
Computation is what happens when geometry relaxes.

%----------------------------------------------------------
\subsection{Self-Reference and Meta-Structure}

Because both programs and values are motifs, RCX-$\pi$ naturally supports
introspection:

\[
\text{classify}(m) \Rightarrow \mu(\langle\text{tag}\rangle,m)
\]

This tagging layer does not change the core, but reveals structure to itself.
It is the beginning of RCX self-awareness.

%----------------------------------------------------------
\subsection{Summary}

\begin{itemize}
\item One primitive constructor: $\mu$
\item Numbers, tuples, and closures are the same thing structurally
\item Programs are motifs that collapse when activated
\item Computation is geometric, not procedural
\item Everything is visible and introspectable as shape
\end{itemize}

RCX-$\pi$ is a universe made of folding. Nothing more is required.
The next sections will expand the evaluator, rules, classifier, and eventually
the seeds of fully self-hosting behavior.