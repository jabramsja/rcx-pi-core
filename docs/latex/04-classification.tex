\section{04 \textbar\ Structural Classification and Meta-Layer}

In RCX-$\pi$, motifs have no inherent ``types.''  
Everything is the same constructor:

\[
\mu(a,b,c,\dots)
\]

Yet when building programs, closures, tuples, numbers,
and later evaluators, we require \emph{structural reflection}.
The classification system provides this.

It is not semantic typing.  
It is \textbf{shape recognition as meaning.}

% ---------------------------------------------------------
\subsection{Why Classification Exists}

A motif can represent:

\begin{itemize}
\item a Peano number
\item a program/closure expecting activation
\item a tuple/pair/triple \dots
\item mixed structures in mid-reduction
\item fully abstract unknown geometry
\end{itemize}

To manipulate them safely we need meta labels.

\[
M \mapsto \langle \text{label}, M \rangle
\]

Labels are grafted \emph{onto the motif itself} in a reversible way.

% ---------------------------------------------------------
\subsection{Core Labels}

The classifier currently distinguishes four canonical categories:

\[
\text{label}(M) \in \{ \text{value}, \text{program}, \text{mixed}, \text{struct} \}
\]

\begin{center}
\begin{tabular}{|c|c|}
\hline
Label & Interpretation \\
\hline
value   & Peano/atomic motif (stable number-like form) \\
program & closure awaiting arguments (callable structure) \\
mixed   & program+data structure not yet reduced \\
struct  & generic motif / composite / unknown form \\
\hline
\end{tabular}
\end{center}

No runtime dispatch.  
No typing discipline.  
Just geometry.

% ---------------------------------------------------------
\subsection{How Tagging Works}

A tagged motif wraps metadata structurally:

\[
\text{tagged}(M) = \mu(\text{meta\_label}, M)
\]

In code, classification is a pure function:

\[
\text{classify}(M) \Rightarrow \text{tagged\_motif}
\]

Example walk:

\[
\text{pair}(2,5)
  \Rightarrow \mu(\mu(\dots),\mu(\dots))\;\text{(raw)}
\]
\[
\text{classify(pair)} 
  \Rightarrow \mu(\langle\text{value}\rangle,\mu(2,5))
\]

Meta-motifs remain motifs. Reduction rules still apply.

This enables future self-reflection loops.

% ---------------------------------------------------------
\subsection{Recognition Rules}

\paragraph{Value Check}
A motif is a numeric value if it matches Peano form:

\[
M = \mu(\mu(\mu(\dots\mu()))) \Rightarrow \text{value}
\]

\paragraph{Program Check}
If the outer layer is one of the known closure patterns:

\[
M = \mu(\text{closure\_pattern},\dots) \Rightarrow \text{program}
\]

\paragraph{Mixed Form}
If partial application exists:

\[
\mu(\text{closure}, x) \text{ unreduced} \Rightarrow \text{mixed}
\]

\paragraph{Fallback}
Everything else is structural:

\[
\_\Rightarrow \text{struct}
\]

All logic is pattern-based.  
There is still no notion of type enforcement.

% ---------------------------------------------------------
\subsection{Why Meta Matters}

The moment a motif can classify another motif,
or itself, the system steps toward self-bootstrap:

\[
M \rightarrow \text{classify}(M)
\]
\[
\text{reduce}(\text{classify}(M)) \rightarrow M'
\]

Self-reflection unlocks:

\begin{itemize}
\item safe program execution restrictions
\item trace inspection
\item printable representation for debugging
\item code that reasons about code
\item later: evaluator written as a motif itself
\end{itemize}

Meta makes RCX-$\pi$ \emph{navigable by itself}.

% ---------------------------------------------------------
\subsection{Example Trace}

\[
M = (2,5) = \mu(2,5)
\]

Classification:

\[
\text{classify}(M)
  \Rightarrow
  \mu(\text{value}, \mu(2,5))
\]

Pretty-printed:

\[
\texttt{<value> (2,5)}
\]

The label is visible, yet evaluation remains purely structural.

% ---------------------------------------------------------
\subsection{Bridge to Pretty Printer}

Classification feeds human-friendly rendering.

Value motifs collapse to numbers,
tuples render as \((a,b,c)\),
tagged structures prefix label annotations.

Pretty printing is not semantic decoration.
It is \textbf{structure translated into readable form.}

% ---------------------------------------------------------
\subsection{Summary}

\begin{itemize}
\item Classification is pattern-based structural tagging
\item Labels describe motifs without altering core rules
\item Supports debugging, printing, safe execution
\item Essential stepping stone toward a self-hosting core
\end{itemize}

RCX-$\pi$ now sees shape.  
Next it will \emph{name} and \emph{speak} it.