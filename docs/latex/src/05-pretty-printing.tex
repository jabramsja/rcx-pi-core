\section{05 \textbar\ Pretty Printing and Structural Rendering}

Raw RCX-$\pi$ motifs are fully structural:

\[
\mu(\mu(\mu(\mu())), \mu(\mu(\mu(\mu(\mu(\mu()))))))
\]

Readable, but only for ascetics.  
Pretty printing is the layer that lets humans \emph{see the motif} as a number,
a pair, a triple, or an $n$-ary tuple.

It does not change meaning.  
It \textbf{interprets shape as form}.

% ---------------------------------------------------------
\subsection{Goals of the Pretty Printer}

\begin{itemize}
\item Render Peano values as natural numbers
\item Display nested motifs as tuples: $(a,b,c,\dots)$
\item Work with or without meta-tags
\item Never break purity of core representation
\item Remain reversible: pretty-printing is view, not mutation
\end{itemize}

Pretty printing is a \emph{lens}.  
The motif underneath is untouched.

% ---------------------------------------------------------
\subsection{Basic Value Rendering}

Given:

\[
0 = \mu(),\quad
1 = \mu(\mu()),\quad
2 = \mu(\mu(\mu()))
\]

The printer collapses Peano depth to an integer:

\[
\texttt{pretty}(\mu(\mu(\mu()))) = 2
\]

Internally:

\[
\text{depth-count}(M) = n \Rightarrow \texttt{int}(n)
\]

This is optional.
Raw structure is still preserved if needed.

% ---------------------------------------------------------
\subsection{Tuples from Nested Motifs}

RCX-$\pi$ treats pairs and triples as plain motifs:

\[
(a,b)    \equiv \mu(a,b)
\]
\[
(a,b,c)  \equiv \mu(a,b,c)
\]

Pretty print becomes:

\[
\texttt{pretty}(\mu(2,5)) = (2,5)
\]
\[
\texttt{pretty}(\mu(2,5,7)) = (2,5,7)
\]

There is no tuple type.  
Tuples are \emph{recognized via arity of structure}.

% ---------------------------------------------------------
\subsection{Higher Arity and Recursion}

The renderer recurses:

\[
\mu(a,b,c,d) \Rightarrow (a,b,c,d)
\]

Each element is printed with value collapse if possible,
otherwise rendered structurally.

Mixed forms remain structural:

\[
\mu(\mu(\mu()), \mu(x,y,z)) \Rightarrow (1,(x,y,z))
\]

% ---------------------------------------------------------
\subsection{Meta-aware Rendering}

Tagged motifs are displayed with label header:

\[
\langle\text{value}\rangle \;\mu(2,5)
  \Rightarrow \texttt{<value> (2,5)}
\]

\[
\langle\text{program}\rangle \;M
  \Rightarrow \texttt{<program> (...)}
\]

Classification adds meaning, printing reflects it.

% ---------------------------------------------------------
\subsection{Raw Structural Form (Fallback)}

Any motif that cannot be recognized prints as canonical form:

\[
\mu(x,y,z)
\]

If no value collapse is available,
\textbf{the printer shows raw geometry}.

This ensures nothing is lost or coerced.

% ---------------------------------------------------------
\subsection{Pretty Printing as Cognitive Tool}

Pretty printing is the difference between

\[
\mu(\mu(\mu(\dots))) 
\]

and

\[
(3,7,12)
\]

It enables:

\begin{itemize}
\item debugging large reductions
\item teaching RCX-$\pi$ concepts visually
\item introspection during self-host development
\item future visualization engines (trees, animations, fold maps)
\end{itemize}

The printer is the first bridge from alien syntax to thought.

% ---------------------------------------------------------
\subsection{Example Session}

Raw:

\[
M = \mu(2,5)
\]

Pretty:

\[
\texttt{pretty}(M) \Rightarrow (2,5)
\]

Classified:

\[
\texttt{pretty}(\text{classify}(M)) \Rightarrow \texttt{<value> (2,5)}
\]

Pretty printer + classifier produce \emph{structural cognition}.

% ---------------------------------------------------------
\subsection{Future Extensions}

\begin{itemize}
\item Color-coded depth maps
\item Pretty-print as tree / graph
\item Unicode glyph bodies ($\mu$-gardens)
\item Interactive fold visualization
\item Reduction animation traces
\item Surface forms for self-host evaluator outputs
\end{itemize}

This module is a seed for UI and introspection tooling.
The machine sees motifs — the printer lets humans see them too.

% ---------------------------------------------------------
\subsection{Summary}

\begin{itemize}
\item Pretty printing converts raw motifs into readable form
\item Works with values, tuples, tagged motifs
\item Does not alter core structure
\item Essential to scaling cognition and debugging
\end{itemize}

RCX-$\pi$ now has a voice.
Soon it will narrate its own execution.