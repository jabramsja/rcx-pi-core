\section{10 \textbar\ Visual Debugging, Structure Tracing, and Shape Introspection}

One of the greatest strengths of RCX-$\pi$ is that computation is not hidden
behind bytecode or control flow. Execution is literally the \emph{reshaping of
a tree}. This chapter formalizes tools for watching that geometry move.

The aim is simple:

\[
\text{computation} = \text{visible structure evolution}
\]

RCX-$\pi$ does not execute instructions; it folds motifs like origami.
A debugger therefore is not a stack tracer, but a \textbf{shape visualizer}.
We will build three progressively richer interfaces:

\begin{enumerate}
\item \textbf{step visualizer:} print reduction steps
\item \textbf{tree renderer:} pretty-print as ASCII shape
\item \textbf{graph visualizer:} export to \texttt{.dot}/Graphviz for render
\end{enumerate}

These tools make the runtime tangible, inspectable, and eventually
\emph{self-reflective}.

%----------------------------------------------------------
\subsection{Reduction Tracing}

The evaluator already reduces motifs through rewrite rules. We extend it with
hooks:

\begin{verbatim}
reduce(m, trace=True)
\end{verbatim}

If tracing is enabled, each reduction step is emitted:

\[
M_0 \Rightarrow M_1 \Rightarrow M_2 \Rightarrow \dots \Rightarrow M_n
\]

We show structural deltas only --- what changed between steps --- to avoid
noise with large expressions.

Example future REPL usage:

\begin{verbatim}
rcx> trace swap 2 5
step 0: μ(swap, μ(2,5))
step 1: μ(5,2)
final:  (5,2)
\end{verbatim}

%----------------------------------------------------------
\subsection{Tree Visualization (ASCII)}

Raw $\mu$-trees can be printed as nested motifs, but visual form makes patterns
obvious. We define a simple ASCII renderer:

\begin{verbatim}
μ
├─ μ
│  └─ μ()
└─ μ
   └─ μ μ μ()
\end{verbatim}

or collapsed tuple form when recognizable:

\[
(2,5,7)
\]

The visualizer becomes essential as we move toward recursion, libraries, and
mutation engines.

%----------------------------------------------------------
\subsection{Graph Rendering (DOT Export)}

We introduce an optional export format:

\begin{verbatim}
rcx> dot swap 2 5 > shape.dot
dot -Tpng shape.dot -o shape.png
\end{verbatim}

Nodes represent motif instances; edges represent children. Visualizing execution
trajectories over time yields \emph{shape films} --- a computational movie.

%----------------------------------------------------------
\subsection{Rewrite Path Maps}

Since RCX computation is geometric, its path through reductions is a graph:

\[
\{M_i\} \text{ with edges } M_i \to M_{i+1}
\]

We store these as directed graphs, enabling:

\begin{itemize}
\item comparison of reduction orders
\item detection of loops, stalls, divergent geometry
\item later: \emph{meta-motifs operating on traces}
\end{itemize}

The engine becomes observable in motion.

%----------------------------------------------------------
\subsection{Lobe and Membrane Visuals (Early Spec)}

Later RCX growth will require structures larger than tuples. We anticipate
\textbf{lobes} --- collections of motifs forming soft semantic clusters --- and
\textbf{membranes} that regulate activation boundaries.

Visual signatures may resemble:

\[
\text{motifs} \to \text{clusters} \to \text{fractal folds}
\]

Early tools will color nodes by classifier tag:

\[
\text{value} / \text{program} / \text{mixed} / \text{struct}
\]

Meta-growth becomes navigable instead of blind.

%----------------------------------------------------------
\subsection{Future Features}

\begin{itemize}
\item animated reduction playback
\item REPL modes: \texttt{:tree}, \texttt{:trace}, \texttt{:graph}
\item zoomable visual maps for large computations
\item mutation overlays to study emergent behavior
\item self-host inspection hooks
\end{itemize}

These tools move RCX-$\pi$ toward a self-aware computational ecology where
shape is not only execution, but \emph{experienceable}.

%----------------------------------------------------------
\subsection{Summary}

\begin{itemize}
\item RCX execution can be visualized --- we make tools to see shape change
\item Tracing reveals reduction paths, not call stacks
\item ASCII + Graphviz give structural insight at multiple scales
\item Lobe/membrane visuals prepare for emergent RCX organisms
\item Debugging becomes watching geometry dance
\end{itemize}

The next chapter designs the mutation engine: from stable logic to evolving
folds.